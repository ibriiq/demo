<<<<<<< HEAD
\documentclass[a4paper]{article}
\usepackage[english]{babel}
\usepackage[utf8]{inputenc}
\usepackage[fleqn]{amsmath}
\usepackage{hyperref}

\begin{document}

\section{Math stuff}

Cubic interpolation for one segment $[x_k, x_{k+1}]$ can be described as:

\begin{equation*}
	\begin{aligned}
		f(t) &= c_{oef1}t^3 + c_{oef2}t^2 + c_{oef3}t + c_{oef4}  \hspace{1cm} with \\
		t(x) &= \frac{x - x_k}{x_{k+1} - x_k}
	\end{aligned}
\end{equation*}

\hspace{1cm} and

\begin{equation*}
	\begin{aligned}
		c_{oef1} &= 2p_0 - 2p_1 - m_0 - m_1 \\
		c_{oef2} &= -3p_0 + 3p_1 - 2m_0 - m_1 \\
		c_{oef3} &= m_0 \\
		c_{oef4} &= p_0
	\end{aligned}
\end{equation*}


(see Wikipedia-Links below)\\
\\
If we rewrite this as function of $d = x - x_k$ we get

\begin{equation*}
	\begin{aligned}
		f'(d) &= c_{oef1}' d^3 + c_{oef2}' d^2 + c_{oef3}' d + c_{oef4}' \hspace{1cm} with \\
		c_{oef1}' &= \frac{c_{oef1}}{(x_{k+1} - x_k)^3} \\
        c_{oef2}' &= \frac{c_{oef2}}{(x_{k+1} - x_k)^2} \\
        c_{oef3}' &= \frac{c_{oef3}}{x_{k+1} - x_k} \\
        c_{oef4}' &= c_{oef4}
    \end{aligned}
\end{equation*}
\\
The implemented algorithm uses two helper variables to calculate the coefficients of $f'$ efficiently:

\begin{equation*}
	\begin{aligned}
		common = m_k + m_{k+1} - 2 \frac{p_{k+1} - p_k}{x_{k+1} - x_k} \\
		invLength = \frac{1}{x_{k+1} - x_k}
    \end{aligned}
\end{equation*}
\\
We use $p_0 = p_k$, $p_1 = p_{k+1}$, $m_0 = m_k (x_{k+1} - x_k)$, $m_1 = m_{k+1} (x_{k+1} - x_k)$ and $s = \frac{p_{k+1}-p_k}{x_{k+1}-x_k}$. The tangents are scaled with the length of the segment. \\
\\
If we insert this into the equations for the coefficients we get the formulas that are used in the algorithm:

\begin{equation*}
\begin{aligned}
	c_{oef1}' &= \frac{c_{oef1}}{(x_{k+1} - x_k)^3} \\
	&= \frac{2p_0 - 2p_1 + m_0 + m_1}{(x_{k+1} - x_k)^3}\\
	&= (2p_k - 2p_{k+1} + m_k (x_{k+1} - x_k) + m_{k+1} (x_{k+1} - x_k) / (x_{k+1} - x_k)^3 \\
	&= \frac {(2p_k - 2p_{k+1} + m_k (x_{k+1} - x_k) + m_{k+1} (x_{k+1} - x_k)}{x_{k+1} - x_k} / (x_{k+1} - x_k)^2 \\
	&= (\frac {2p_k - 2p_{k+1}}{x_{k+1} - x_k} + m_k + m_{k+1}  ) * invLength^2 \\
	&= (-2\frac {p_{k+1}- p_k}{x_{k+1} - x_k} + m_k + m_{k+1}  ) * invLength^2 \\
	&= common * invLenght^2
\end{aligned}
\end{equation*}
 
\begin{equation*}
\begin{aligned}
	c_{oef2}' &= \frac{c_{oef2}}{(x_{k+1} - x_k)^2} \\
	&= (-3p_0 + 3p_1 - 2m_0 - m_1) / (x_{k+1} - x_k)^2 \\
	&= (-3p_k + 3p_{k+1} - 2* m_k (x_{k+1} - x_k) - m_{k+1} (x_{k+1} - x_k)) / (x_{k+1} - x_k)^2 \\
	&= (\frac{-3p_k + 3p_{k+1}}{x_{k+1} - x_k} - 2m_k - m_{k+1}) * invLenght \\
	&= (3\frac{p_{k+1} - p_k}{x_{k+1} - x_k} - 2m_k - m_{k+1}) * invLenght \\
	&=(\frac{p_{k+1} - p_k}{x_{k+1} - x_k} + 2\frac{p_{k+1} - p_k}{x_{k+1} - x_k} - m_k - m_{k+1} - m_k) * invLenght \\
	&= (s - common - m_k) * invLenght  
\end{aligned}    
\end{equation*}
 
\begin{equation*}
\begin{aligned}
	c_{oef3}' &= \frac{c_{oef3}}{x_{k+1} - x_k} \\
	&= \frac{m_0}{x_{k+1} - x_k} \\
	&= \frac{m_k (x_{k+1} - x_k)}{x_{k+1} - x_k} \\
 	&= m_k
\end{aligned}
\end{equation*}

\begin{equation*}
\begin{aligned}
	c_{oef4}' &= c_{oef4} = p_0 = p_k
\end{aligned}
\end{equation*}

\section{Useful Links}

\url{http://de.wikipedia.org/w/index.php?title=Kubisch_Hermitescher_Spline&oldid=130168003)}\\
\url{http://en.wikipedia.org/w/index.php?title=Monotone_cubic_interpolation&oldid=622341725}\\
\url{http://math.stackexchange.com/questions/45218/implementation-of-monotone-cubic-interpolation}\\
\url{http://math.stackexchange.com/questions/4082/equation-of-a-curve-given-3-points-and-additional-constant-requirements#4104}

=======
\documentclass[a4paper]{article}
\usepackage[english]{babel}
\usepackage[utf8]{inputenc}
\usepackage[fleqn]{amsmath}
\usepackage{hyperref}

\begin{document}

\section{Math stuff}

Cubic interpolation for one segment $[x_k, x_{k+1}]$ can be described as:

\begin{equation*}
	\begin{aligned}
		f(t) &= c_{oef1}t^3 + c_{oef2}t^2 + c_{oef3}t + c_{oef4}  \hspace{1cm} with \\
		t(x) &= \frac{x - x_k}{x_{k+1} - x_k}
	\end{aligned}
\end{equation*}

\hspace{1cm} and

\begin{equation*}
	\begin{aligned}
		c_{oef1} &= 2p_0 - 2p_1 - m_0 - m_1 \\
		c_{oef2} &= -3p_0 + 3p_1 - 2m_0 - m_1 \\
		c_{oef3} &= m_0 \\
		c_{oef4} &= p_0
	\end{aligned}
\end{equation*}


(see Wikipedia-Links below)\\
\\
If we rewrite this as function of $d = x - x_k$ we get

\begin{equation*}
	\begin{aligned}
		f'(d) &= c_{oef1}' d^3 + c_{oef2}' d^2 + c_{oef3}' d + c_{oef4}' \hspace{1cm} with \\
		c_{oef1}' &= \frac{c_{oef1}}{(x_{k+1} - x_k)^3} \\
        c_{oef2}' &= \frac{c_{oef2}}{(x_{k+1} - x_k)^2} \\
        c_{oef3}' &= \frac{c_{oef3}}{x_{k+1} - x_k} \\
        c_{oef4}' &= c_{oef4}
    \end{aligned}
\end{equation*}
\\
The implemented algorithm uses two helper variables to calculate the coefficients of $f'$ efficiently:

\begin{equation*}
	\begin{aligned}
		common = m_k + m_{k+1} - 2 \frac{p_{k+1} - p_k}{x_{k+1} - x_k} \\
		invLength = \frac{1}{x_{k+1} - x_k}
    \end{aligned}
\end{equation*}
\\
We use $p_0 = p_k$, $p_1 = p_{k+1}$, $m_0 = m_k (x_{k+1} - x_k)$, $m_1 = m_{k+1} (x_{k+1} - x_k)$ and $s = \frac{p_{k+1}-p_k}{x_{k+1}-x_k}$. The tangents are scaled with the length of the segment. \\
\\
If we insert this into the equations for the coefficients we get the formulas that are used in the algorithm:

\begin{equation*}
\begin{aligned}
	c_{oef1}' &= \frac{c_{oef1}}{(x_{k+1} - x_k)^3} \\
	&= \frac{2p_0 - 2p_1 + m_0 + m_1}{(x_{k+1} - x_k)^3}\\
	&= (2p_k - 2p_{k+1} + m_k (x_{k+1} - x_k) + m_{k+1} (x_{k+1} - x_k) / (x_{k+1} - x_k)^3 \\
	&= \frac {(2p_k - 2p_{k+1} + m_k (x_{k+1} - x_k) + m_{k+1} (x_{k+1} - x_k)}{x_{k+1} - x_k} / (x_{k+1} - x_k)^2 \\
	&= (\frac {2p_k - 2p_{k+1}}{x_{k+1} - x_k} + m_k + m_{k+1}  ) * invLength^2 \\
	&= (-2\frac {p_{k+1}- p_k}{x_{k+1} - x_k} + m_k + m_{k+1}  ) * invLength^2 \\
	&= common * invLenght^2
\end{aligned}
\end{equation*}
 
\begin{equation*}
\begin{aligned}
	c_{oef2}' &= \frac{c_{oef2}}{(x_{k+1} - x_k)^2} \\
	&= (-3p_0 + 3p_1 - 2m_0 - m_1) / (x_{k+1} - x_k)^2 \\
	&= (-3p_k + 3p_{k+1} - 2* m_k (x_{k+1} - x_k) - m_{k+1} (x_{k+1} - x_k)) / (x_{k+1} - x_k)^2 \\
	&= (\frac{-3p_k + 3p_{k+1}}{x_{k+1} - x_k} - 2m_k - m_{k+1}) * invLenght \\
	&= (3\frac{p_{k+1} - p_k}{x_{k+1} - x_k} - 2m_k - m_{k+1}) * invLenght \\
	&=(\frac{p_{k+1} - p_k}{x_{k+1} - x_k} + 2\frac{p_{k+1} - p_k}{x_{k+1} - x_k} - m_k - m_{k+1} - m_k) * invLenght \\
	&= (s - common - m_k) * invLenght  
\end{aligned}    
\end{equation*}
 
\begin{equation*}
\begin{aligned}
	c_{oef3}' &= \frac{c_{oef3}}{x_{k+1} - x_k} \\
	&= \frac{m_0}{x_{k+1} - x_k} \\
	&= \frac{m_k (x_{k+1} - x_k)}{x_{k+1} - x_k} \\
 	&= m_k
\end{aligned}
\end{equation*}

\begin{equation*}
\begin{aligned}
	c_{oef4}' &= c_{oef4} = p_0 = p_k
\end{aligned}
\end{equation*}

\section{Useful Links}

\url{http://de.wikipedia.org/w/index.php?title=Kubisch_Hermitescher_Spline&oldid=130168003)}\\
\url{http://en.wikipedia.org/w/index.php?title=Monotone_cubic_interpolation&oldid=622341725}\\
\url{http://math.stackexchange.com/questions/45218/implementation-of-monotone-cubic-interpolation}\\
\url{http://math.stackexchange.com/questions/4082/equation-of-a-curve-given-3-points-and-additional-constant-requirements#4104}

>>>>>>> b72c5ea86bf6885e302b926b7202718cf917849d
\end{document}